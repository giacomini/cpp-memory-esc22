% !TeX root = cpp-esc22.tex

\section{Resource management}

\begin{frame}[fragile]{Weaknesses of a \texttt{T*}}

  \begin{itemize}[<+->]
  \item Critical information is not encoded in the type
    \only<.>{
      \begin{itemize}[<.>]
      \item Am I the owner of the pointee? Should I delete it?
      \item Is the pointee an object or an array of objects? of what size?
      \item Was it allocated with \texttt{new}, \texttt{malloc} or even something
        else (e.g. \texttt{fopen} returns a \texttt{FILE*})?
      \end{itemize}
    }
  \item Owning pointers are prone to leaks and double deletes
  \item Owning pointers are unsafe in presence of exceptions
  \item Runtime overhead
    \begin{itemize}
    \item<.-> dynamic allocation/deallocation
    \item<.-> indirection
    \end{itemize}
  \end{itemize}

  \begin{onlyenv}<1> % \only doesn't work for a verbatim env
    \begin{codeblock}
T* p = create\_something();\end{codeblock}
  \end{onlyenv}

  \begin{onlyenv}<2>
    \begin{codeblock}
\{
  T* p = new T\{\};
  \ddd
  // ops, forgot to delete p
\}
\{
  T* p = new T;
  \ddd
  delete p;
  \ddd
  delete p; // ops, delete again
\}\end{codeblock}
  \end{onlyenv}

  \begin{onlyenv}<3>
    \begin{codeblock}
\{
  T* p = new T;
  \ddd // potentially throwing code
  delete p;
\}\end{codeblock}
  \end{onlyenv}

\end{frame}

\begin{frame}[fragile]{Debugging memory problems}

  \begin{itemize}
  \item Valgrind is a suite of debugging and profiling tools for memory
    management, threading, caching, etc.
  \item Valgrind Memcheck can detect
    \begin{itemize}
    \item invalid memory accesses
    \item use of uninitialized values
    \item memory leaks
    \item bad frees
    \end{itemize}
  \item It's precise, but slow
  \end{itemize}

  \begin{onlyenv}<2>
    \begin{codeblock}
$ g++ leak.cpp
$ valgrind ./a.out
==18331== Memcheck, a memory error detector
...\end{codeblock}
  \end{onlyenv}

\end{frame}

\begin{frame}[fragile]{Debugging memory problems \insertcontinuationtext}

  \begin{itemize}
  \item \textit{Address Sanitizer} (ASan)
  \item The compiler instruments the executable so that at runtime ASan can
    catch problems similar, but not identical, to valgrind
  \item Faster than valgrind
  \end{itemize}

  \begin{onlyenv}<2>
    \begin{codeblock}
$ g++ -fsanitize=address leak.cpp
$ ./a.out

=================================================================
==18338==ERROR: LeakSanitizer: detected memory leaks
\ldots\end{codeblock}
  \end{onlyenv}

\end{frame}

\begin{frame}{Hands-on}
  \begin{itemize}
  \item C++ $\rightarrow$ Memory issues
  \item Get familiar with Valgrind and memory sanitizers
  \item Inspect, compile, run directly and run through valgrind or memory
    sanitizers (not both together)
    \begin{itemize}
    \item \code{non\_owning\_pointer.cpp}
    \item \code{array\_too\_small.cpp}
    \item \code{leak.cpp}
    \item \code{double\_delete.cpp}
    \item \code{missed\_delete.cpp}
    \end{itemize}
  \item Try and fix the problems
  \end{itemize}
\end{frame}

\begin{frame}{When to use a \code{T*}}
  \begin{itemize}
  \item To represent a \textit{link} to an object when
    \begin{itemize}
    \item the object is not owned, and
    \item the link may be null or the link can be re-bound
    \end{itemize}
  \item Mutable and immutable scenarios
    \begin{itemize}
    \item \code{T*} vs \code{T const*}
    \end{itemize}
  \end{itemize}
\end{frame}

\begin{frame}[fragile]{When \underline{not} to use a \code{T*}}
  \begin{itemize}
  \item To represent a link to an object when
    \begin{itemize}
    \item the object is owned, or
    \item the link can never be null, and the link cannot be re-bound
    \end{itemize}
  \item Alternatives
    \begin{itemize}
    \item use a copy
    \item use a (const) reference
      \begin{codeblock}
T& tr = t1;  // tr is an alias for t1
tr = t2;     // doesn't re-bind tr, assigns t2 to t1

T* tp = &t1; // tp points to t1
tp = &t2;    // re-binds tp, it now points to t2\end{codeblock}
    \item use a resource-managing object
      \begin{itemize}
      \item \code{std::array}, \code{std::vector}, \code{std::string},
        \textit{smart pointers}, \ldots
      \end{itemize}
    \end{itemize}
  \end{itemize}
\end{frame}

\begin{frame}{Resource management}
  \begin{itemize}
  \item Dynamic memory is just one of the many types of resources manipulated by a
    program:
    \begin{itemize}
    \item thread, mutex, socket, file, \ldots
    \end{itemize}
  \item C++ offers powerful tools to manage resources
    \begin{itemize}
    \item \textit{"C++ is my favorite garbage collected language because it
      generates so little garbage"}
    \end{itemize}
  \end{itemize}

\end{frame}

\begin{frame}{Smart pointers}
  \begin{itemize}[<+->]
  \item Objects that behave like pointers, but also manage the lifetime of the pointee
  \item Leverage the RAII idiom
    \begin{itemize}[<.->]
    \item Resource Acquisition Is Initialization
    \item Resource (e.g. memory) is acquired in the constructor
    \item Resource (e.g. memory) is released in the destructor
    \end{itemize}
  \item Importance of how the destructor is designed in C++
    \begin{itemize}[<.->]
    \item deterministic: guaranteed execution at the end of the scope
    \item order of execution opposite to order of construction
    \end{itemize}
  \item Guaranteed no leak nor double release, even in presence of exceptions
  \end{itemize}
\end{frame}

\begin{frame}[fragile]{Smart pointers \insertcontinuationtext}
  \begin{codeblock}
template<typename Pointee>
class SmartPointer \{
  Pointee* m\_p;
 public:
  explicit SmartPointer(Pointee* p): m\_p\{p\} \{\}
  \~SmartPointer() \{ delete m\_p; \}
  \visible<3->{Pointee* operator\alert<3>{->}() \{ return m\_p; \}
  Pointee& operator\alert<3>{*}() \{ return *m\_p; \}}
  \visible<4->{\ddd}
\};

class Histo \{ \ddd \};

\{
  SmartPointer<Histo> sp\{new Histo\{\}\};
  \visible<2->{sp\alert<2-3>{->}fill();
  (\alert<2-3>{*}sp).fill();}
\}\end{codeblock}

At the end of the scope (i.e. at the closing \code{\}}) \code{sp} is destroyed
and its destructor \code{delete}s the pointee

\end{frame}

\begin{frame}[fragile]{\texttt{std::unique\_ptr<T>}}

  Standard smart pointer

  \begin{itemize}
  \item Exclusive ownership
  \item No space nor time overhead
  \item Non-copyable, movable
  \end{itemize}

  \begin{codeblock}<2->{
class Histo \{ \ddd \};

void take(std::unique\_ptr<Histo> ph); \uncover<6>{    \alert{// by value}}

\uncover<3->{std::unique\_ptr<Histo> ph\{new Histo\{\}\};   // explicit new}
\uncover<4->{auto ph = std::make\_unique<Histo>();      // better}
\uncover<5->{take(ph);}                                 \uncover<6->{// error, non-copyable}
\uncover<7->{take(\alert<7>{std::move(}ph\alert<7>{)});}                      \uncover<8->{// ok, movable}}\end{codeblock}

\uncover<9->{NB: \code{std::move} doesn't actually move anything. It just signals to the compiler that it's ok to move
the object}
\end{frame}

\begin{frame}[fragile]{\texttt{std::shared\_ptr<T>}}

  Standard smart pointer

  \begin{itemize}
  \item Shared ownership (reference counted)
  \item Some space and time overhead
    \begin{itemize}
    \item for the management, not for access
    \end{itemize}
  \item Copyable and movable
  \end{itemize}

  \begin{codeblock}<2->{
class Histo \{ \ddd \};

void take(std::shared\_ptr<Histo> px);

\uncover<3->{std::shared\_ptr<Histo> ph\{new Histo\{\}\}; // explicit new}
\uncover<4->{auto px = std::make\_shared<Histo>();    // better}
\uncover<5->{take(px);}                               \uncover<6->{// ok, copyable}
\uncover<7->{take(std::move(px));}                    \uncover<8->{// ok, movable}}\end{codeblock}

\end{frame}

\begin{frame}[fragile]{Using smart pointers}

    \begin{itemize}[<+->]

    \item Give an owning raw pointer (e.g. the result of a call to \code{new})
      to a smart pointer as soon as possible
    \item Prefer \code{unique\_ptr} unless you need \code{shared\_ptr}
      \begin{itemize}[<.->]
      \item You can always move a \code{unique\_ptr} into a \code{shared\_ptr}
      \item But not viceversa
      \end{itemize}

    \item Access to the raw pointer is available
      \begin{itemize}[<.->]
      \item e.g. to pass to legacy APIs
      \item \code{\textit{smart}\_ptr<T>::get()}
        \begin{itemize}
        \item returns a \textbf{non-owning} \code{T*}
        \end{itemize}
      \item \code{unique\_ptr<T>::release()}
        \begin{itemize}
        \item returns an \textbf{owning} \code{T*}
        \item must be explicitly managed
        \end{itemize}
      \end{itemize}

    \item Arrays are supported

      \begin{codeblock}<.->
std::unique_ptr<int[]> p\{new int[n]\}; // destructor calls \upquote{delete []}\end{codeblock}

    \end{itemize}

\end{frame}

\begin{frame}[fragile]{\texttt{\textit{smart}\_ptr} and functions}

  Pass a smart pointer to a function only if the function needs to rely on
  the smart pointer itself

  \begin{itemize}

  \item<2-> by value of a \texttt{unique\_ptr}, to transfer ownership
    \begin{codeblock}
void take(std::unique\_ptr<Histo> u);
auto u = std::make\_unique<Histo>();
take(u);            // error
take(std::move(u)); // ok\end{codeblock}

\item<3-> by value of a \texttt{shared\_ptr}, to keep the resource alive
  \begin{codeblock}
auto s = std::make\_shared<Histo>();
std::thread t\{[=] \{ do\_something\_with(s); \}\};\end{codeblock}

\item<4-> by reference, to interact with the smart pointer itself
  \begin{codeblock}
void print\_count(std::shared\_ptr<Histo> const& s) \{
  std::cout <{}< s.use\_count() <{}< '\\n';
\};
auto s = std::make\_shared<Histo>();
print\_count(s);\end{codeblock}

  \end{itemize}

\end{frame}

\begin{frame}[fragile]{\texttt{\textit{smart}\_ptr} and functions \insertcontinuationtext}
  \begin{itemize}
  \item Otherwise pass the pointee by (const) reference/pointer
    \begin{codeblock}<2->
\uncover<3->{void fill(std::shared\_ptr<Histo> s) \{ if (s) s->fill(); \}}
\uncover<4->{void fill(Histo* t)                 \{ if (t) t->fill(); \} // better}
\uncover<5->{void fill(Histo& t)                 \{ t.fill(); \}         // better}

auto s = make\_shared<Histo>();
\uncover<3->{fill(s);}
\uncover<4->{fill(s.get());}
\uncover<5->{if (s) fill(*s);}\end{codeblock}

  \item<6-> Return a \texttt{\textit{smart}\_ptr} from a function if the
    function has dynamically allocated a resource that is passed to the caller

    \begin{codeblock}
auto factory() \{ return std::make\_unique<Histo>(); \}

\uncover<7->{auto u = factory();     // std::unique\_ptr<Histo>
std::shared\_ptr<Histo> s = std::move(u);}

\uncover<8->{std::shared\_ptr<Histo> s = factory();}\end{codeblock}

  \end{itemize}

\end{frame}

\begin{frame}[fragile]{\texttt{\textit{smart}\_ptr} custom deleter}
  \begin{itemize}
  \item \texttt{\textit{smart}\_ptr} is a general-purpose resource handler
  \item The resource release is not necessarily done with \texttt{delete}
  \item \texttt{unique\_ptr} and \texttt{shared\_ptr} support a
    \textit{custom deleter}
  \end{itemize}
  \begin{columns}[t]
    \begin{column}{.5\textwidth}
      \begin{codeblock}<2->{
FILE* f = std::fopen(\ddd);
\ddd
std::fclose(f);}\end{codeblock}

      \uncover<3->{Usual problems:
      \begin{itemize}
      \item Who owns the resource?
      \item Forgetting to release
      \item Releasing twice
      \item Early return/throw
      \end{itemize}}

    \end{column}

    \begin{column}{.5\textwidth}
      \begin{codeblock}<4->{
auto f = std::shared\_ptr<FILE>\{
  std::fopen(\ddd),
  [](auto p) \{ std::fclose(p); \}
\};}\end{codeblock}

      \begin{itemize}
      \item<5-> Wrap the deallocation function in a lambda to be safe in presence of
        multiple overloads
      \item<6-> A bit more involved for \code{unique\_ptr}
      \end{itemize}
    \end{column}

  \end{columns}
\end{frame}

\begin{frame}{Hands-on}
  \begin{itemize}
  \item C++ $\rightarrow$ Memory issues
    \begin{itemize}
    \item Adapt the exercises to use smart pointers, when applicable
    \item Remember to compile with \code{-fsanitize=address}
    \end{itemize}
  \item C++ $\rightarrow$ Managing resources
  \item Starting from \code{dir.cpp} and following the hints in the file, write
    code to:
    \begin{itemize}
    \item create a smart pointer managing a \code{DIR} resource obtained with
      the \code{opendir} function call
    \item associate a deleter to that smart pointer
    \item implement a function to read the names of the files in that directory
    \item check if the deleter is called at the right moment
    \item hide the creation of the smart pointer behind a factory function
    \item populate a vector of FILEs, properly wrapped in a smart pointer,
      obtained opening the regular files in that directory
    \item \ldots
    \end{itemize}
  \end{itemize}
\end{frame}

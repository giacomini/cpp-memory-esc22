% !TeX root = cpp-esc22.tex

\section{Move semantics}

\begin{frame}[fragile]{We can do better than copying}

  \begin{columns}
    \begin{column}{.45\textwidth}

      \begin{codeblock}{\tiny
class String \{
  char* s\_;
  \ddd
\};}\end{codeblock}

      \begin{codeblock}<1->{\tiny
String s1\{"Hi!"\};
\uncover<2->{String s2\{s1\};}}\end{codeblock}

  \uncover<3->{\small
    \begin{itemize}
    \item Both \code{s1} and \code{s2} exist at the end
    \item The ``deep'' copy is needed
    \end{itemize}
  }

  \begin{codeblock}<4->{\tiny
String get\_string() \{ return "Hi!"; \}
String \alert<6>{s3}\{\alert<5>{get\_string()}\}\alert<7>{;}}\end{codeblock}

  \uncover<8->{\small
    \begin{itemize}
    \item Only \code{s3} exists at the end
    \item The ``deep'' copy is a waste
    \end{itemize}
  }
    \end{column}

    \begin{column}{.55\textwidth}
      \newcommand*{\stackx}{0.}
      \newcommand*{\heapx}{3.0}

      \begin{tikzpicture}
        [anchor=south west]
        \node at (\stackx,0) [
          rectangle,
          minimum width=2cm,
          minimum height=7cm,
          draw=black,
          label={90:{\scriptsize stack}}
        ] {};
        \node at (\heapx,0) [
          rectangle,
          minimum width=2cm,
          minimum height=7cm,
          draw=black,
          label={90:{\scriptsize heap}}
        ] {};
        \node (s1remote) at (\heapx-0.1,5.4) [
          rectangle,
          fill=green!10!white,
          draw=black!40,
          minimum width=2.2cm,
          minimum height=0.7cm,
        ] {};
        \node at (\heapx,5.5) [
          rectangle,
          minimum width=2cm,
          minimum height=.5cm,
          draw=black,
          fill=green!30!white
        ] {\scriptsize\tt Hi!\bslash{0}};
        \node at (\heapx,5.5) [
          rectangle,
          minimum width=.5cm,
          minimum height=.5cm,
          draw=black,
          densely dotted,
          label={270:{\tiny\tt 0x4abc}}
        ] {};
        \node at (\stackx-0.1,5.4) [
          rectangle,
          fill=green!10!white,
          draw=black!40,
          minimum width=2.2cm,
          minimum height=.7cm,
          label={180:{\scriptsize\tt s1}},
        ] {};
        \node (s1pointer) at (\stackx,5.5) [
          rectangle,
          minimum width=2cm,
          minimum height=.5cm,
          draw=black,
          fill=green!30!white,
        ] {\scriptsize\tt 0x4abc};  
        \draw[->] (s1pointer.east) -- (s1remote.west);
        \visible<2->{
          \node (s2remote) at (\heapx-0.1,3.9) [
            rectangle,
            fill=green!10!white,
            draw=black!40,
            minimum width=2.2cm,
            minimum height=0.7cm,
          ] {};
          \node at (\heapx,4) [
            rectangle,
            minimum width=2cm,
            minimum height=.5cm,
            draw=black,
            fill=green!30!white
          ] {\scriptsize\tt Hi!\bslash{0}};
          \node at (\heapx,4) [
            rectangle,
            minimum width=.5cm,
            minimum height=.5cm,
            draw=black,
            densely dotted,
            label={270:{\tiny\tt 0x4ab0}}
          ] {};
          \node at (\stackx-0.1,3.9) [
            rectangle,
            fill=green!10!white,
            draw=black!40,
            minimum width=2.2cm,
            minimum height=.7cm,
            label={180:{\scriptsize\tt s2}},
          ] {};
          \node (s2pointer) at (\stackx,4) [
            rectangle,
            minimum width=2cm,
            minimum height=.5cm,
            draw=black,
            fill=green!30!white,
          ] {\scriptsize\tt 0x4ab0};
          \draw[->] (s2pointer.east) -- (s2remote.west);
        }
        \visible<2>{\path[black!70!white] (s1remote.south) -- (s2remote.north) node[yshift=.25cm,single arrow, draw, shape border rotate=270, single arrow tip angle=120,single arrow head extend=.1cm,minimum height=.5cm,minimum width=1cm] {};
        }
        \visible<5-6>{
          \node (tmpremote) at (\heapx-0.1,1.9) [
            rectangle,
            fill=green!10!white,
            draw=black!40,
            minimum width=2.2cm,
            minimum height=0.7cm,
          ] {};
          \node at (\heapx,2) [
            rectangle,
            minimum width=2cm,
            minimum height=.5cm,
            draw=black,
            fill=green!30!white
          ] {\scriptsize\tt Hi!\bslash{0}};
          \node at (\heapx,2) [
            rectangle,
            minimum width=.5cm,
            minimum height=.5cm,
            draw=black,
            densely dotted,
            label={270:{\tiny\tt 0x4aa0}}
          ] {};
          \node at (\stackx-0.1,1.9) [
            rectangle,
            fill=green!10!white,
            draw=black!40,
            minimum width=2.2cm,
            minimum height=.7cm,
          ] {};
          \node (tmppointer) at (\stackx,2) [
            rectangle,
            minimum width=2cm,
            minimum height=.5cm,
            draw=black,
            fill=green!30!white,
          ] {\scriptsize\tt 0x4aa0};
          \draw[->] (tmppointer) -- (tmpremote);  
        }
        \visible<6->{
          \node (s3remote) at (\heapx-0.1,.4) [
            rectangle,
            fill=green!10!white,
            draw=black!40,
            minimum width=2.2cm,
            minimum height=0.7cm,
          ] {};
          \node at (\heapx,.5) [
            rectangle,
            minimum width=2cm,
            minimum height=.5cm,
            draw=black,
            fill=green!30!white
          ] {\scriptsize\tt Hi!\bslash{0}};
          \node at (\heapx,.5) [
            rectangle,
            minimum width=.5cm,
            minimum height=.5cm,
            draw=black,
            densely dotted,
            label={270:{\tiny\tt 0x4aac}}
          ] {};
          \node at (\stackx-0.1,.4) [
            rectangle,
            fill=green!10!white,
            draw=black!40,
            minimum width=2.2cm,
            minimum height=.7cm,
            label={180:{\scriptsize\tt s3}},
          ] {};
          \node (s3pointer) at (\stackx,.5) [
            rectangle,
            minimum width=2cm,
            minimum height=.5cm,
            draw=black,
            fill=green!30!white,
          ] {\scriptsize\tt 0x4aac};
          \draw[->] (s3pointer) -- (s3remote);
        }
        \visible<6>{\path[black!70!white] (tmpremote.south) -- (s3remote.north) node[yshift=.25cm,single arrow, draw, shape border rotate=270, single arrow tip angle=120,single arrow head extend=.1cm,minimum height=.5cm,minimum width=1cm] {};
        }
      \end{tikzpicture}
    \end{column}
  \end{columns}

\end{frame}

\begin{frame}[fragile]{Copy vs move}

  \begin{columns}
    \begin{column}{.45\textwidth}

      \begin{codeblock}{\tiny
\uncover<1->{class String \{
  char* s\_;
 public:
  String(char const* s) \{
    size\_t size = strlen(s) + 1;
    s\_ = new char[size];
    memcpy(s\_, s, size);
  \}
  ~String() \{ delete [] s\_; \}}
\uncover<2->{  // \alert<2>{copy}
  String(String const& other) \{
    size\_t size = strlen(other.s\_) + 1;
    s\_ = new char[size];
    memcpy(s\_, other.s\_, size);
  \}}
\uncover<4->{  // \alert<4-5>{move}
  String(\alert<7>{???} tmp): s\_(tmp.s\_) \{
\uncover<5->{    tmp.s\_ = nullptr;}
  \}}
\uncover<1->{  \ddd
\};

String s1\{"Hi!"\};}
\uncover<2->{String s2\{s1\};}
\uncover<3->{String \alert<4>{s3}\{\alert<3>{get\_string()}\}\alert<6>{;}}
}\end{codeblock}

    \end{column}

    \begin{column}{.55\textwidth}
      \newcommand*{\stackx}{0.}
      \newcommand*{\heapx}{3.0}

      \begin{tikzpicture}
        [anchor=south west]
        \node at (\stackx,0) [
          rectangle,
          minimum width=2cm,
          minimum height=7cm,
          draw=black,
          label={90:{\scriptsize stack}}
        ] {};
        \node at (\heapx,0) [
          rectangle,
          minimum width=2cm,
          minimum height=7cm,
          draw=black,
          label={90:{\scriptsize heap}}
        ] {};
        \node (s1remote) at (\heapx-0.1,5.4) [
          rectangle,
          fill=green!10!white,
          draw=black!40,
          minimum width=2.2cm,
          minimum height=0.7cm,
        ] {};
        \node at (\heapx,5.5) [
          rectangle,
          minimum width=2cm,
          minimum height=.5cm,
          draw=black,
          fill=green!30!white
        ] {\scriptsize\tt Hi!\bslash{0}};
        \node at (\heapx,5.5) [
          rectangle,
          minimum width=.5cm,
          minimum height=.5cm,
          draw=black,
          densely dotted,
          label={270:{\tiny\tt 0x4abc}}
        ] {};
        \node at (\stackx-0.1,5.4) [
          rectangle,
          fill=green!10!white,
          draw=black!40,
          minimum width=2.2cm,
          minimum height=.7cm,
          label={180:{\scriptsize\tt s1}},
        ] {};
        \node (s1pointer) at (\stackx,5.5) [
          rectangle,
          minimum width=2cm,
          minimum height=.5cm,
          draw=black,
          fill=green!30!white,
        ] {\scriptsize\tt 0x4abc};
        \draw[->] (s1pointer.east) -- (s1remote.west);
        \visible<2->{
          \node at (\heapx-0.1,3.9) [
            rectangle,
            fill=green!10!white,
            draw=black!40,
            minimum width=2.2cm,
            minimum height=0.7cm,
          ] {};
          \node at (\heapx,4) [
            rectangle,
            minimum width=2cm,
            minimum height=.5cm,
            draw=black,
            fill=green!30!white
          ] {\scriptsize\tt Hi!\bslash{0}};
          \node at (\heapx,4) [
            rectangle,
            minimum width=.5cm,
            minimum height=.5cm,
            draw=black,
            densely dotted,
            label={270:{\tiny\tt 0x4ab0}}
          ] (r2) {};
          \node at (\stackx-0.1,3.9) [
            rectangle,
            fill=green!10!white,
            draw=black!40,
            minimum width=2.2cm,
            minimum height=.7cm,
            label={180:{\scriptsize\tt s2}},
          ] {};
          \node (s2pointer) at (\stackx,4) [
            rectangle,
            minimum width=2cm,
            minimum height=.5cm,
            draw=black,
            fill=green!30!white,
          ] (h2) {\scriptsize\tt 0x4ab0};
          \draw[->] (s2pointer.east) -- (s2remote.west);
        }
        \visible<2>{\path[black!70!white] (s1remote.south) -- (s2remote.north) node[yshift=.25cm,single arrow, draw, shape border rotate=270, single arrow tip angle=120,single arrow head extend=.1cm,minimum height=.5cm,minimum width=1cm] {};
        }
        \visible<3->{
          \node (tmpremote) at (\heapx-0.1,1.9) [
            rectangle,
            fill=green!10!white,
            draw=black!40,
            minimum width=2.2cm,
            minimum height=0.7cm,
          ] {};
          \node at (\heapx,2) [
            rectangle,
            minimum width=2cm,
            minimum height=.5cm,
            draw=black,
            fill=green!30!white
          ] {\scriptsize\tt Hi!\bslash{0}};
          \node at (\heapx,2) [
            rectangle,
            minimum width=.5cm,
            minimum height=.5cm,
            draw=black,
            densely dotted,
            label={270:{\tiny\tt 0x4aa0}}
          ] {};
        }
        \visible<3-5>{
          \node at (\stackx-0.1,1.9) [
            rectangle,
            fill=green!10!white,
            draw=black!40,
            minimum width=2.2cm,
            minimum height=.7cm,
          ] {};
          \node (tmppointer) at (\stackx,2) [
            rectangle,
            minimum width=2cm,
            minimum height=.5cm,
            draw=black,
            fill=green!30!white,
          ] {\scriptsize\tt \alt<3-4>{\alert<4>{0x4aa0}}{\code{\alert<5>{nullptr}}}};
        }
        \visible<3-4>{\draw[->] (tmppointer.east) -- (tmpremote.west);}
        \visible<4->{
          \node at (\stackx-0.1,.4) [
            rectangle,
            fill=green!10!white,
            draw=black!40,
            minimum width=2.2cm,
            minimum height=.7cm,
            label={180:{\scriptsize\tt s3}},
          ] {};
          \node (s3pointer) at (\stackx,.5) [
            rectangle,
            minimum width=2cm,
            minimum height=.5cm,
            draw=black,
            fill=green!30!white,
          ] {\scriptsize\tt \alert<4>{0x4aa0}};
          \draw[->] (s3pointer.east) -- ([yshift=-.5ex]tmpremote.west);
        }
        \visible<4>{\path[black!70!white] (tmppointer.south) -- (s3pointer.north) node[yshift=.3cm,xshift=-.1cm,single arrow, draw, shape border rotate=270, single arrow tip angle=120,single arrow head extend=.1cm,minimum height=.5cm,minimum width=.4cm] {};
        }
      \end{tikzpicture}
    \end{column}
  \end{columns}
\end{frame}

\begin{frame}{lvalues vs rvalues}
  \begin{itemize}
  \item The taxonomy of values in C++ is complex
    \begin{itemize}
    \item glvalue, prvalue, xvalue, lvalue, rvalue
    \end{itemize}
  \item We can assume
    \begin{description}
    \item [lvalue] A named object
      \begin{itemize}
      \item for which you can take the address
      \item \alert{l} stands for ``left'' because it used to represent the
        \alert{l}eft-hand side of an assignment
      \end{itemize}
    \item [rvalue] An unnamed (temporary) object
      \begin{itemize}
      \item for which you can't take the address
      \item \alert{r} stands for ``right'' because it used to represent the
        \alert{r}ight-hand side of an assignment
      \end{itemize}
    \end{description}
  \end{itemize}
\end{frame}

\begin{frame}[fragile]{Rvalue reference}
  \begin{itemize}
  \item A \textbf{\code{T\&\&}} is an rvalue reference
    \begin{itemize}
    \item introduced in C++11
    \end{itemize}
  \item It binds to rvalues but not to lvalues
  \end{itemize}

  \begin{codeblock}<2->{\tiny
class String \{
  // copy constructor
  \alert<3>{String(String const& other)} \{ \ddd \}
  // move constructor
  \alert<4>{String(String\alert<2>{&&} tmp)} \{ \ddd \}
\};

\uncover<3->{String s2\{s1\};           // call String::String(String const&)}
\uncover<4->{String s3\{get_string()\}; // call String::String(String&&)}}\end{codeblock}

\end{frame}

\begin{frame}[fragile]{Special member functions}

  \begin{itemize}
  \item A class has five special member functions
    \begin{itemize}
    \item Plus the default constructor
    \end{itemize}

    \begin{codeblock}{\tiny
class Widget \{
  Widget(Widget const&);            // copy constructor
  Widget& operator=(Widget const&); // copy assignment
  Widget(Widget&&);                 // move constructor
  Widget& operator=(Widget&&);      // move assignment
  \~Widget();                        // destructor
\};}\end{codeblock}

  \item<2-> The compiler can generate them automatically according to some
    convoluted rules
    \begin{itemize}
    \item The behavior depends on the behavior of data members
    \end{itemize}
  \item<3-> Rules of thumb
    \begin{description}
    \item [Rule of zero] Don't declare them and rely on the compiler
    \item [Rule of five] If you need to declare one, declare them all
    \end{description}
    \begin{itemize}
    \item Consider \code{= default} and \code{= delete}
    \end{itemize}
  \end{itemize}

\end{frame}

\begin{frame}{Hands-on}
  
  \begin{itemize}
  \item C++ $\rightarrow$ Move operations 
  \item Open the program \code{string.cpp} and complete the existing code to:

    \begin{itemize}
    \item Complete the set of the special member functions so that \code{String} is copyable and movable
    \item Instead of a raw pointer, keep a \code{unique\_ptr} in the private part of String
    \item \ldots
    \end{itemize}

\end{itemize}

\end{frame}

\begin{frame}[fragile]{Return a value from a function}
  \begin{itemize}
  \item Returning a large value from a function is often perceived as slow
    \begin{itemize}

    \item<2-> Return ``by pointer''
      \begin{codeblock}
std::unique\_ptr<LargeObject> make\_large\_object() \{
  return std::make\_unique<LargeObject>();
\}

auto lo = make\_large\_object();
lo->\ddd; // use the object, via a pointer\end{codeblock}

     \item<3-> Use ``out'' arguments
       \begin{codeblock}
void make\_large\_object(LargeObject& o) \{
  o = LargeObject\{\};   // requires copy assignment
\}

LargeObject lo;        // requires default constructor
make\_large\_object(lo);
lo.\ddd                 // use the object\end{codeblock}

    \end{itemize}
  \end{itemize}

\end{frame}

\begin{frame}[fragile]{Return a value from a function \insertcontinuationtext}

  \begin{itemize}
  \item There are very few reasons for not doing the obvious
    \begin{codeblock}
LargeObject make\_large\_object() \{
  return LargeObject\{\};
\}

auto lo = make\_large\_object(); // possibly auto const
lo.\ddd                         // use the object\end{codeblock}

  \item In fact the compiler is allowed or even obliged in some circumstances to
    elide the copy of the returned value into the final destination
    \begin{itemize}
    \item (N)RVO -- (Named) Return Value Optimization
    \end{itemize}
  \item If (N)RVO is not applied, a move is done, if available
  \item If the move is not available, copy
  \end{itemize}
\end{frame}

\begin{frame}[fragile]{Return value optimization}

  \begin{columns}[t]

    \begin{column}{.5\textwidth}
      Unnamed
      \begin{codeblock}
Widget make\_widget()
\{
  if (\ddd) \{
    return Widget\{\};
  \}
  return Widget\{\};
\}

auto w = make\_widget();\end{codeblock}
    \end{column}

    \begin{column}{.5\textwidth}
      Named

      \begin{codeblock}
Widget make\_widget()
\{
  Widget result;
  if (\ddd) \{
    result = Widget\{\};
  \}
  return result;
\}

auto w = make\_widget();\end{codeblock}
    \end{column}
  \end{columns}

  \begin{itemize}[<2->]
  \item Try not to mix named and unamed \code{return}s in the same function
  \item Avoid \code{return std::move(result)}, unless necessary
  \end{itemize}
\end{frame}

\begin{frame}{Hands-on}

  \begin{itemize}
  \item C++ $\rightarrow$ Return Value Optimization
  \item Open the program \code{rvo.cpp}. Implement variations of the
    \code{make_vector} function so that:
    \begin{itemize}
    \item the result is returned from the function
    \item the result is passed to the function as an output parameter (by
      reference or by pointer)
    \end{itemize}

  \item Measure the time it takes to execute them. Discuss the results.
  \end{itemize}
\end{frame}

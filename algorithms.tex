% !TeX root = cpp-esc22.tex

\section{Algorithms and functions}

\begin{frame}{The C++ standard library}

  \begin{itemize}[<+->]
  \item The standard library contains components of general use
    \begin{itemize}[<.->]
    \item \alert<2>{containers (data structures)}
    \item \alert<2>{algorithms}
    \item strings
    \item input/output
    \item random numbers
    \item regular expressions
    \item concurrency and parallelism
    \item filesystem
    \item ...
    \end{itemize}

  \item The subset containing containers and algorithms is
    known as STL (Standard Template Library)
  \item But templates are everywhere
  \end{itemize}

\end{frame}

\begin{frame}{STL Algorithms}

  \begin{itemize}[<1->]
  \item Generic functions that operate on \alert{ranges} of objects
  \item Implemented as function templates
  \end{itemize}

  \pause

  \begin{description}[<+->]
    \item[Non-modifying] \code{all\_of any\_of for\_each
      count count\_if mismatch equal find find\_if adjacent\_find search ...}
    \item[Modifying] \code{copy fill generate transform remove
      replace swap reverse rotate shuffle sample unique ...}
    \item[Partitioning] \code{partition stable\_partition ...}
    \item[Sorting] \code{sort partial\_sort nth\_element ... }
    \item[Set] \code{set\_union set\_intersection set\_difference
      ...}
    \item[Min/Max] \code{min max minmax lexicographical\_compare
      clamp ...}
    \item[Numeric] \code{iota accumulate inner\_product partial\_sum
      adjacent\_difference ...}
  \end{description}
\end{frame}

\begin{frame}{Range}

  \begin{itemize}[<+->]
  \item A range is defined by a pair of \alert{iterators}
    [\textit{first}, \textit{last}), with \textit{last} referring to one
    past the last element in the range
    \begin{itemize}
    \item the range is \textit{half-open}
    \item \textit{first} == \textit{last} means the range is empty
    \item \textit{last} can be used to return failure
    \end{itemize}
  \item An \alert{iterator} allows to go through the elements of the associated
    range
    \begin{itemize}
    \item operations to advance, access, compare
    \item typically obtained from containers calling specific methods
    \end{itemize}
  \item An iterator is a generalization of a pointer
    \begin{itemize}
    \item it supports the same operations, possibly through overloaded operators
    \item certainly \code{* ++ -> == !=}, maybe \code{-{}- + - += -= <}
    \end{itemize}
  \item C++20 introduced \code{ranges}, a new library of \textit{concepts} and components for dealing with ranges of
    objects (not discussed here)
  \end{itemize}

\end{frame}

\begin{frame}[fragile]{Range \insertcontinuationtext}

  \begin{tikzpicture}
    [anchor=south west]
    \visible<1->{\node at (0,0) [memory] {};}
    \visible<1->{\node at (0,0) [word anchor,label={90:{\tiny\tt 0x0000}}] {};}
    \visible<1->{\node at (.9\textwidth-.25cm,0) [word anchor,label={90:{\tiny\tt 0xffff}}] {};}

    \visible<1->{\node at (3.9,-0.1) [
        rectangle,
        fill=green!10!white,
        draw=black!40,
        minimum width=3.2cm,
        minimum height=0.7cm,
        label=270:{\scriptsize\tt a},
      ] {};
    }
    \visible<1->{\node (a0) at (4,0) [word] {\scriptsize\tt\alert<5>{123}};}
    \visible<1->{\node (a1) at (5,0) [word] {\scriptsize\tt\alert<8>{456}};}
    \visible<1->{\node (a2) at (6,0) [word] {\scriptsize\tt\alert<11>{789}};}
    \visible<1->{\node (a3) at (7,0) [phantom word] {};}
    \visible<1->{\node at (4,0) [word anchor,label={90:{\tiny\tt 0xab00}}] {};}
    \visible<1->{\node at (5,0) [word anchor,label={90:{\tiny\tt 0xab04}}] {};}
    \visible<1->{\node at (6,0) [word anchor, label={90:{\tiny\tt 0xab08}}] {};}
    \visible<1->{\node at (7,0) [word anchor, label={90:{\tiny\tt0xab0c}}] {};}
    \visible<2->{\node (first) at (4,-1.5) [word, label={270:{\scriptsize\tt first}}] {};}
    \visible<3->{\node (last) at (7,-1.5) [word, label={270:{\scriptsize\tt last}}] {};}
    \visible<2-5>{\draw[->] (first.north) -- (a0.south);}
    \visible<3->{\draw[->] (last.north) --  (a3.south);}
    \visible<6-8>{\draw[->] (first.north) -- (a1.south);}
    \visible<9-11>{\draw[->] (first.north) -- (a2.south);}
    \visible<12->{\draw[->] (first.north) -- (a3.south);}

  \end{tikzpicture}

  \begin{codeblock}
\uncover<1->{std::array\invisible<16>{<int,3>} a = \{123, 456, 789\};\uncover<16>{ // \alert{CTAD}}}
\uncover<2->{\alert<2>{auto first = a.begin();}       // or std::begin(a)}
\uncover<3->{\alert<3>{auto\only<14->{\alert<14>{ const}} last = a.end();}\only<-13>{      }    // or std::end(a)}
\uncover<4->{while (\alert<4,7,10,13>{first != last}) \{       // comparison
  ... \alert<5,8,11>{*first} ...;             // access
  \alert<6,9,12>{++first};                    // advance
\}}\end{codeblock}

  \begin{itemize}[<15->]
  \item \code{std::array<T>::iterator} models the \textit{RandomAccessIterator} concept
  \end{itemize}
\end{frame}

\begin{frame}[fragile]{Generic programming}

  \begin{itemize}[<+->]
  \item A style of programming in which \alert{algorithms} are written
    in terms of \alert{concepts}

  \begin{codeblock}<+->
template <class Iterator, class T>
Iterator
find(Iterator first, Iterator last, const T\& value)
\{
  for (; first \alert{!=} last; \alert{++}first)
    if (\alert{*}first \alert{==} value)
      break;
  return first;
\}\end{codeblock}

  \item A concept is a set of requirements that a type needs to satisfy
    \begin{itemize}[<.->]
    \item e.g. supported expressions, nested types, memory layout, \ldots
    \end{itemize}
  \end{itemize}


\end{frame}

\begin{frame}[fragile]{Range \insertcontinuationtext}

  \begin{tikzpicture}
    [anchor=south west]
    \visible<1->{\node at (0,0) [memory] {};}
    \visible<1->{\node at (0,0) [word anchor,label={90:{\tiny\tt 0x0000}}] {};}
    \visible<1->{\node at (.9\textwidth-.25cm,0) [word anchor,label={90:{\tiny\tt 0xffff}}] {};}

    \visible<1->{\node (l) at (0.9,-0.1) [
        rectangle,
        fill=green!10!white,
        draw=black!40,
        minimum width=1.2cm,
        minimum height=0.7cm,
        label=270:{\scriptsize\tt l},
      ] {};
    }
    \visible<1->{\node (a0) at (3.5,0) [word] {\scriptsize\tt\alert<5>{123}};}
    \visible<1->{\node (a1) at (6.5,0) [word] {\scriptsize\tt\alert<8>{456}};}
    \visible<1->{\node (a2) at (5  ,0) [word] {\scriptsize\tt\alert<11>{789}};}
    \visible<1->{\node (a3) at (8  ,0) [phantom word,draw=black,densely dotted] {};}
    \visible<1->{\node (c0) at (3.5,0) [word anchor,label={90:{\tiny\tt 0xab00}}] {};}
    \visible<1->{\node (c1) at (6.5,0) [word anchor,label={90:{\tiny\tt 0xad08}}] {};}
    \visible<1->{\node (c2) at (5,0) [word anchor, label={90:{\tiny\tt 0xac04}}] {};}
    \visible<1->{\node (c3) at (8,0) [word anchor] {};}
    \visible<1->{\draw[Circle-Stealth,black!50] (l.east) .. controls +(1,1) .. (c0.north);}
    \visible<1->{\draw[Circle-Stealth,black!50] (a0.east) .. controls +(1,1.5) .. (c1.north);}
    \visible<1->{\draw[Circle-Stealth,black!50] (a1.east) .. controls +(-1,1) .. (c2.north);}
    \visible<1->{\draw[Circle-Stealth,black!50] (a2.east) .. controls +(1,1.5) .. (c3.north);}
    %    \visible<1->{\node at (3,0) [word anchor,label={90:{\tiny\tt 0xab00}}] {};}
    \visible<2->{\node (first) at (3.5,-1.5) [word, label={270:{\scriptsize\tt first}}] {};}
    \visible<3->{\node (last) at (8,-1.5) [word, label={270:{\scriptsize\tt last}}] {};}
    \visible<2-5>{\draw[->] (first.north) -- (a0.south);}
    \visible<3->{\draw[->] (last.north) --  (a3.south);}
    \visible<6-8>{\draw[->] (first.north) -- (a1.south);}
    \visible<9-11>{\draw[->] (first.north) -- (a2.south);}
    \visible<12->{\draw[->] (first.north) -- (a3.south);}

  \end{tikzpicture}

  \begin{codeblock}
\uncover<1->{std::forward\_list<int> l = \{123, 456, 789\};}
\uncover<2->{\alert<2>{auto first = l.begin();}}
\uncover<3->{\alert<3>{auto const last = l.end();}}
\uncover<4->{while (\alert<4,7,10,13>{first != last}) \{
  ... \alert<5,8,11>{*first} ...;
  \alert<6,9,12>{++first};
\}}\end{codeblock}

  \begin{itemize}[<14->]
  \item \code{std::forward\_list<T>::iterator} models the \textit{ForwardIterator} concept
  \end{itemize}
\end{frame}

\begin{frame}[fragile]{Algorithms and ranges}

  \begin{itemize}
  \item Examples
    \begin{codeblock}
std::vector v = \{ 23, 54, 41, 0, 18 \};

// sort the vector in ascending order
std::sort(std::begin(v), std::end(v));

// sum up the vector elements, initializing the sum to 0
auto s = std::accumulate(std::begin(v), std::end(v), 0);
auto r = std::reduce(std::begin(v), std::end(v));

// append the partial sums of the vector elements into a list
std::list<int> l;
std::partial\_sum(std::begin(v), std::end(v), std::back\_inserter(l));

// find the first element with value 42
auto it = std::find(std::begin(v), std::end(v), 42);
    \end{codeblock}

  \item Some algorithms are customizable passing a function
    \begin{codeblock}
auto it = std::find_if(v.begin(), v.end(), \alert{\textit{filter}});\end{codeblock}
  \end{itemize}
\end{frame}

\begin{frame}{Hands-on}
  \begin{itemize}
  \item C++ $\rightarrow$ Algorithms
  \item Starting from \code{algo.cpp} and following the hints, write code to
    \begin{itemize}
    \item sum all the elements of the vector
    \item compute the average of the first half and of the second half of the
      vector
    \item move the three central numbers to the beginning
    \item remove duplicate elements
    \item \ldots
    \end{itemize}
  \end{itemize}
\end{frame}

\begin{frame}[fragile]{Why using standard algorithms}

  \begin{itemize}[<+->]
  \item They are correct
  \item They express intent more clearly than a raw \code{for} loop
  \item They enable \textit{parallelism for the masses}
    \begin{itemize}[<.->]
    \item parallel algorithms available in C++17, implementations are appearing
    \end{itemize}
  \end{itemize}

  \begin{codeblock}<+->{
#include <execution>

std::vector<int> v = \ddd;
std::sort(\alert{std::execution::par}, v.begin(), v.end());
auto it = std::find(\alert{std::execution::par}, v.begin(), v.end(), 42);}\end{codeblock}

\end{frame}

\begin{frame}{Hands-on}
  \begin{itemize}
  \item C++ $\rightarrow$ Algorithms
  \item Starting from \code{algo\_par.cpp} and following the hints, write code to
    \begin{itemize}
    \item sum all the elements of the vector, with and without parallelization
    \item sort the vector, with and without parallelization
    \item \ldots
    \end{itemize}
    and compare the execution times.
  \end{itemize}
\end{frame}

\begin{frame}[fragile]{Functions}

  \begin{itemize}
  \item<1-> A function associates a sequence of statements (the function
    \textit{body}) with a name and a list of zero or more parameters

  \begin{codeblock}<2->
\alt<2>{std::string}{\alert<3>{auto        }} join(std::string const\& s, int i)
\{
  return s + \upquote{-} + std::to\_string(i);
\}

join("XYZ", 5); // std::string\{"XYZ-5"\}\end{codeblock}

  \item<4-> A function may return a value
  \item<5-> A function returning a \code{bool} is called a \textit{predicate}
    \begin{codeblock}
\alert<5>{bool} less(int n, int m) \{ return n < m; \}\end{codeblock}

  \item<6-> Multiple functions can have the same name $\rightarrow$
    \textit{overloading}
    \begin{itemize}
    \item different parameter lists
    \end{itemize}
  \end{itemize}

\end{frame}

\begin{frame}[fragile]{Algorithms and functions}

  \begin{codeblock}{\tiny
template <class Iterator, class T>
Iterator find(Iterator first, Iterator last, \alert{const T\& value})
\{
  for (; first != last; ++first)
    if (\alert{*first == value})
      break;
  return first;
\}

auto it = find(v.begin(), v.end(), 42);}\end{codeblock}

  \begin{codeblock}<2->
template <class Iterator, class T>
Iterator find_if(Iterator first, Iterator last, \alert{T pred})
\{
  for (; first != last; ++first)
    if (\alert{pred(*first)})          // unary predicate
      break;
  return first;
\}

\uncover<3->{bool lt42(int n) \{ return n < 42; \}}

\uncover<4->{auto it = find_if(v.begin(), v.end(), lt42);}\end{codeblock}

\end{frame}

\begin{frame}[fragile]{Function objects}

  A mechanism to define \textit{something-callable-like-a-function}
  \begin{itemize}
  \item<4-> A class with an \texttt{operator()}
  \end{itemize}

  \begin{columns}[t]

    \begin{column}{.5\textwidth}

      \begin{codeblock}<2->{
\invisible{struct LessThan42 \{}
auto \alert<2,4>{lt42}(int n)
\{
  return n < 42;
\}

\invisible{
LessThan42 lt42\{\};}
auto b = \alert<2,7>{lt42(}32\alert<2,7>{)}; // true

\uncover<3->{vector v\{61,32,51\};
auto it = find_if(
    begin(v), end(v),
    \alert<3,8>{lt42}
); // *it == 32}}\end{codeblock}

    \end{column}

    \begin{column}{.5\textwidth}

      \begin{codeblock}<4->{
\alert<4>{struct LessThan42 \{}
  auto \alert<4>{operator()}(int n) \alert<4>{const}
  \{
    return n < 42;
  \}
\alert<4>{\};}

\uncover<5->{\alt<5>{LessThan42 lt42\{\};}{auto lt42 = LessThan42\{\};}}
\uncover<7->{auto b = \alert<7>{lt42(}32\alert<7>{)}; // true}

\uncover<8,9>{vector v\{61,32,51\};
auto it = find_if(
    begin(v), end(v),
    \alert{\alt<8>{lt42}{LessThan42\{\}}}
); // *it == 32}}\end{codeblock}

    \end{column}
  \end{columns}
\end{frame}

\begin{frame}[fragile]{Function objects \insertcontinuationtext}
  A function object, being the instance of a class, can have state
  \begin{codeblock}<2->{
class LessThan \{
  \alert<2>{int m\_;}
 public:
  explicit \alert<2>{LessThan(int m) : m\_\{m\} \{\}}
  auto operator()(int n) const \{
    return n < m_;
  \}
\};

\uncover<3->{auto lt42 = LessThan\{42\};
auto b1 = \alt<-5>{lt42}{\alert<6>{LessThan\{42\}}}(32); // true}

\uncover<4->{auto lt24 = LessThan\{24\};
auto b2 = \alt<-5>{lt24}{\alert<6>{LessThan\{24\}}}(32); // false}

\uncover<5->{vector v\{61,32,51\};
auto i1 = find_if(\ldots, \alt<-5>{lt42}{\alert<6>{LessThan\{42\}}}); // *i1 == 32
auto i2 = find_if(\ldots, \alt<-5>{lt24}{\alert<6>{LessThan\{24\}}}); // i2 == end, i.e. not found}}\end{codeblock}

  \uncover<6>{}

\end{frame}

\begin{frame}[fragile]{Function objects \insertcontinuationtext}
  An example from the standard library

  \begin{codeblock}
#include <random>

// random bit generator
std::default_random_engine gen;

// generate N 32-bit unsigned integer numbers
for (int n = 0; n != N; ++n) \{
  std::cout <{}< \alert{gen()} <{}< \bslashn;
\}

// generate N floats distributed normally (mean: 0., stddev: 1.)
std::normal_distribution<float> dist;
for (int n = 0; n != N; ++n) \{
  std::cout <{}< \alert{dist(gen)} <{}< \bslashn;
\}

// generate N ints distributed uniformly between 1 and 6 included
std::uniform\_int\_distribution<> roll\_dice(1, 6);
for (int n = 0; n != N; ++n) \{
  std::cout <{}< \alert{roll\_dice(gen)} <{}< \bslashn;
\}\end{codeblock}

\end{frame}

\begin{frame}[fragile]{Lambda expression}

  \begin{itemize}
  \item A concise way to create an unnamed function object
  \item Useful to pass actions/callbacks to algorithms, threads, frameworks,
    \ldots
  \end{itemize}

  \begin{columns}[t]

    \begin{column}{.5\textwidth}
      \begin{codeblock}<2->{
struct LessThan42 \{
  auto operator()(int n)
  \{
    return n < 42;
  \}
\};

class LessThan \{
  int m\_;
 public:
  explicit LessThan(int m)
    : m\_\{m\} \{\}
  auto operator()(int n) const
  \{
    return n < m_;
  \}
\};}\end{codeblock}

    \end{column}

    \begin{column}{.5\textwidth}
      \begin{codeblock}<2->{
find_if(\ldots, LessThan42\{\});

\uncover<3->{find_if(\ldots, \alert<3>{[](int n) \{}
               return n < 42;
             \alert<3>{\}}
);}

find_if(\ldots, LessThan\{42\});

\uncover<4->{\only<4>{\alert{int m\{42\};}}
find_if(\ldots, [\alert<4->{\only<4>{=}\only<5>{m = 42}}](int n) \{
              return n < \alert<4->{m};
             \}
);}}\end{codeblock}
    \end{column}

  \end{columns}
\end{frame}

\begin{frame}[fragile]{Lambda expression \insertcontinuationtext}
  The evaluation of a lambda expression produces an unnamed function object (a
  \textit{closure})
  \begin{itemize}
  \item The \texttt{operator()} corresponds to the code of the body of the
    lambda expression
  \item The data members are the captured local variables

    \begin{itemize}
    \item \texttt{[]} capture nothing
    \item \texttt{[=]} capture all by value
    \item \texttt{[k]} capture \texttt{k} by value
    \item \texttt{[\&]} capture all by reference
    \item \texttt{[\&k]} capture \texttt{k} by reference
    \item \texttt{[=, \&k]} capture all by value but \texttt{k} by reference
    \item \texttt{[\&, k]} capture all by reference but \texttt{k} by value
    \end{itemize}

  \item Global variables are available without being captured
  \end{itemize}
\end{frame}

\begin{frame}{Hands-on}
  \begin{itemize}
  \item C++ $\rightarrow$ Algorithms
  \item Starting from \code{algo\_functions.cpp} and following the hints, write
    code to
    \begin{itemize}
    \item multiply all the elements of the vector
    \item sort the vector in descending order
    \item move the even numbers to the beginning
    \item create another vector with the squares of the numbers in the first vector
    \item find the first multiple of 3 or 7
    \item erase from the vector all the multiples of 3 or 7
    \item \ldots
    \end{itemize}
  \end{itemize}
\end{frame}

\begin{frame}[fragile]{\texttt{std::function}}
  \begin{itemize}
  \item Type-erased wrapper that can store and invoke any callable entity with a certain signature
    \begin{itemize}
    \item function, function object, lambda, member function
    \end{itemize}
  \item Some space and time overhead, so use only if a template parameter is not satisfactory
  \end{itemize}

  \begin{codeblock}<2->{
#include <functional>
#include <iostream>

\alert{int} sum\_squares\alert{(int} x\alert{, int} y\alert{)} \{ return x * x + y * y; \}

int main() \{
  std::vector<std::function<\alert{int(int, int)}>{}> v \{
    std::plus<>\{\},       // has a compatible operator()
    std::multiplies<>\{\}, // idem
    \&sum\_squares
  \};
  for (int k = 10; k <= 1000; k *= 10) \{
    v.push\_back([k]\alert{(int} x\alert{, int} y\alert{)} -> \alert{int} \{ return k * x + y; \});
  \}
  for (auto const\& f : v) \{ std::cout <{}< f(4, 5) <{}< \bslashn; \}
\}}\end{codeblock}

\end{frame}

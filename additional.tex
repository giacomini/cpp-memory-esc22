% !TeX root = cpp-esc22.tex

\section{Additional material}

\section*{Type deduction}

\begin{frame}[fragile]{\texttt{auto}}
  Let the compiler deduce the type of a variable from the initializer

  \begin{codeblock}
\uncover<1->{auto i = 0;                // int
auto u = 0U;               // unsigned int
auto p = &i;               // int*
auto d = 1.;               // double
auto c = \upquote{a};              // char
auto s = "a";              // char const*}
\uncover<2->{auto t = std::string\{"a"\}; // std::string
std::vector<std::string> v;
auto it = std::begin(v);   // std::vector<std::string>::iterator}
\uncover<3->{using namespace std::chrono\_literals;
auto u = 1234us;           // std::chrono::microseconds}
\uncover<4->{auto e;                    // error}\end{codeblock}
\end{frame}

\begin{frame}[fragile]{\texttt{auto} and references}

  \begin{itemize}
  \item \texttt{auto} never deduces a reference
  \item if needed, \code{\&} must be added explicitly
  \end{itemize}

\begin{codeblock}
T v;

auto  v1 = v;     // T  -- v1 is a copy of v
auto\& v2 = v;     // T\& -- v2 is an alias of v
\alert{auto  v3 = v2;    // T  -- v2 is a copy of v}\end{codeblock}

\end{frame}

\begin{frame}[fragile]{\code{auto} and \code{const}}

  \begin{itemize}
  \item \code{auto} makes a mutable copy
  \item \code{auto const} (or \code{const auto}) makes a non-mutable copy
  \item \code{auto\&} preserves \code{const}-ness
  \end{itemize}

  \begin{codeblock}
T v;

auto        v1 = v; // T        -- v1 is a mutable copy of v
auto const  v2 = v; // T const  -- v2 is a non-mutable copy of v
auto\&       v3 = v; // \alert{T\&}       -- v3 is a \alert{mutable} alias of v
auto const\& v4 = v; // T const\& -- v4 is a non-mutable alias of v\end{codeblock}

  \begin{codeblock}
T const v;

auto        v1 = v; // T        -- v1 is a mutable copy of v
auto const  v2 = v; // T const  -- v2 is a non-mutable copy of v
auto\&       v3 = v; // \alert{T const\&} -- v3 is a \alert{non-mutable} alias of v
auto const\& v4 = v; // T const\& -- v4 is a non-mutable alias of v\end{codeblock}

\end{frame}

\begin{frame}[fragile]{How to check the deduced type?}
  \begin{itemize}
  \item Trick by S. Meyers
    \begin{codeblock}{
template<typename T> struct D;

auto k = 0U;
D<decltype(k)> d; // error: aggregate 'D<\alert{unsigned int}> d'...

auto const o = 0.;
D<decltype(o)> d; // error: aggregate 'D<\alert{const double}> d'...

auto const\& f = 0.f;
D<decltype(f)> d; // error: aggregate 'D<\alert{const float\&}> td'...

auto s = "hello";
D<decltype(s)> d; // error: aggregate 'D<\alert{const char*}> d'...

auto\& t = "hello";
D<decltype(t)> d; // error: aggregate 'D<\alert{const char (\&)[6]}> d'...}\end{codeblock}

  \item \code{decltype} returns the type of an expression
    \begin{itemize}
    \item at compile time
    \end{itemize}
  \end{itemize}
\end{frame}

\section*{More on lambda expressions}

\begin{frame}[fragile]{Lambda: closure}
  The evaluation of a lambda expression produces an unnamed function object (a
  \textit{closure})
  \begin{itemize}
  \item The \texttt{operator()} corresponds to the code of the body of the
    lambda expression
  \item The data members are the captured local variables
  \end{itemize}

  \begin{columns}[t]
    \begin{column}{.5\textwidth}
      \begin{codeblock}<2->{
\uncover<4-6>{int v = 3;}

\uncover<2->{auto l = [}\temporal<5-6>{}{=}{v = 3}\uncover<2->{]}\uncover<3->{(\alt<-7>{int }{auto} i)\uncover<9->{\alert<9>{ -> int}}
\{ return i + v; \}}

\uncover<6->{auto r = l(5); // 8}}\end{codeblock}

      \uncover<7>{}
    \end{column}

    \begin{column}{.5\textwidth}
      \begin{codeblock}<2->{
\uncover<2->{class SomeUniqueName \{
  \uncover<5->{int v\_;}
 public:
  \uncover<5->{explicit SomeUniqueName(int v)
    : v\_\{v\} \{\}}
  \uncover<8->{template<typename T>}
  \alt<-8>{auto}{\alert<9>{int }} operator()\uncover<3->{(\alt<-7>{int}{T  } i)
  \{ return i + v\uncover<5->{\_}; \}}
\};}

\uncover<4->{int v = 3;}
\uncover<2->{auto l = SomeUniqueName\{}\uncover<5->{v}\uncover<2->{\};}
\uncover<6->{auto r = l(5); // 8}}\end{codeblock}
    \end{column}
  \end{columns}
\end{frame}

\begin{frame}[fragile]{Lambda: capturing}

  \begin{itemize}
  \item Automatic variables used in the body need to be captured
    \begin{itemize}
    \item \texttt{[]} capture nothing
    \item \texttt{[=]} capture all by value
    \item \texttt{[k]} capture \texttt{k} by value
    \item \texttt{[\&]} capture all by reference
    \item \texttt{[\&k]} capture \texttt{k} by reference
    \item \texttt{[=, \&k]} capture all by value but \texttt{k} by reference
    \item \texttt{[\&, k]} capture all by reference but \texttt{k} by value
    \end{itemize}

    \begin{columns}[t]
      \begin{column}{.5\textwidth}
        \begin{codeblock}<2->{
int v = 3;
auto l = [\only<3->{\alert<3>{\&}}v] \{\};}\end{codeblock}
      \end{column}

      \begin{column}{.5\textwidth}
        \begin{codeblock}<2->{
class SomeUniqueName \{
  int\only<3->{\alert<3>{\&}} v\_;
 public:
  explicit SomeUniqueName(int\only<3->{\alert<3>{\&}} v)
    : v\_\{v\} \{\}
  \ddd
\};

auto l = SomeUniqueName\{v\};}\end{codeblock}
      \end{column}
    \end{columns}
  \item<4-> Global variables are available without being captured
  \end{itemize}
\end{frame}

\begin{frame}[fragile]{Lambda: \texttt{const} and \texttt{mutable}}
  \begin{itemize}
  \item By default the call to a lambda is \texttt{const}
    \begin{itemize}
    \item Variables captured by value are not modifiable
    \end{itemize}
  \item<2-> A lambda can be declared \texttt{mutable}
  \end{itemize}

  \begin{columns}
    \begin{column}{.5\textwidth}
      \begin{codeblock}<1->{
[]\only<2->{\alert<2>{() mutable}}\only<3->{\alert<3>{ -> void}} \{\};}\end{codeblock}
    \end{column}

    \begin{column}{.5\textwidth}
      \begin{codeblock}<1->{
struct SomeUniqueName \{
  \alt<-2>{auto}{\alert<3>{void}} operator()() \only<1>{\alert{const} }\{\}
\};}\end{codeblock}
    \end{column}
  \end{columns}

  \begin{itemize}
  \item<4-> Variables captured by reference can be modified
    \begin{itemize}
    \item There is no way to capture by \texttt{const\&}
    \end{itemize}
  \end{itemize}

  \begin{codeblock}<4->
int v = 3;
[\&v] \{ ++v; \}();
assert(v == 4);\end{codeblock}

\end{frame}

\begin{frame}[fragile]{Lambda: dangling reference}

  Be careful not to have dangling references in a closure
  \begin{itemize}
  \item It's similar to a function returning a reference to a local variable
  \end{itemize}

  \begin{codeblock}
auto make\_lambda()
\{
  int v = 3;
  return [\alert{&}] \{ return v; \}; // return a closure
\}

auto l = make\_lambda();
auto d = l(); // the captured variable is dangling here\end{codeblock}

  \begin{codeblock}
auto start\_in\_thread()
\{
  int v = 3;
  return std::async([\alert{&}] \{ return v; \});
\}\end{codeblock}

\end{frame}


\section*{More on move}

\begin{frame}[fragile]{Rvalue reference}
  \begin{itemize}
  \item A \textbf{\code{T\&\&}} is an rvalue reference
    \begin{itemize}
    \item introduced in C++11
    \end{itemize}
  \item It binds to rvalues but not to lvalues
  \end{itemize}

  \begin{codeblock}<2->{\tiny
class Thing;
Thing make\_thing();

Thing t;

\uncover<3->{Thing      &  r = t;}\uncover<4->{            // ok}
\uncover<5->{Thing      && r = t;}\uncover<6->{            // error}
\uncover<7->{Thing      &  r = make\_thing();}\uncover<8->{ // error}
\uncover<9->{Thing      && r = make\_thing();}\uncover<10->{ // ok}
\uncover<11->{Thing const&  r = make\_thing();}\uncover<12->{ // ok (!)}
\uncover<13->{Thing const&& r = make\_thing();}\uncover<14->{ // ok, but what for?}}\end{codeblock}

  \begin{codeblock}<15->{\tiny
class String \{
  // move constructor
  String(String\alert<15>{&&} tmp) : s\_(tmp.s\_) \{
    tmp.s\_ = nullptr;
  \}
\};

String s2\{s1\};           // call String::String(String const&)
String s3\{get\_string()\}; // call String::String(String&&)}\end{codeblock}

\end{frame}

\begin{frame}[fragile]{Rvalue reference \insertcontinuationtext}
  \begin{itemize}
  \item Any function can accept rvalue references
    \begin{codeblock}
void foo(String&&);

foo(get\_string());
foo(String\{"hello"\});\end{codeblock}
  \item lvalues can be explicitly transformed into rvalues
    \begin{codeblock}
String s;
foo(s);            // error
foo(std::move(s)); // ok, I don't care any more about s
s.size();          // dangerous\end{codeblock}
  \end{itemize}

\end{frame}

\begin{frame}[fragile]{Overloading on \&\&}
  \begin{itemize}
  \item<1-> A function can be overloaded for temporaries
    \begin{itemize}
    \item useful if there are significant opportunities of optimization
    \end{itemize}
    \begin{codeblock}{\tiny
void foo(Widget const&) \{\ddd\}
void foo(Widget&&) \{\ddd\}

Widget w\{\ddd\};
foo(w);           // calls foo(Widget const&)
foo(Widget\{\ddd\}); // calls foo(Widget&&)}\end{codeblock}
  \item<2-> For more than one parameter it becomes less desirable
    \begin{itemize}
    \item consider pass by value, if move is cheap
    \item especially useful for "sinks", e.g. in constructors
      \begin{codeblock}{\tiny
struct S \{
  T1 t1\_; T2 t2\_;
  S(T1 t1, T2 t2) : t1\_(std::move(t1)), t2\_(std::move(t2)) \{\ddd\}
\};

T1 t1; T2 t2;
S s\{t1, make\_t2()\};
S s\{make\_t1(), t2\};}\end{codeblock}
    \end{itemize}
  \end{itemize}
\end{frame}
\begin{frame}[fragile]{Copy operations}

  \begin{codeblock}{\tiny
class Widget \{
  \ddd
  Widget(Widget const& other);
  Widget& operator=(Widget const& other);
\};}\end{codeblock}

  \begin{description}[<2->]
  \item [copy constructor] Allows the \alert{construction} of an object as a copy of another object

    \begin{codeblock}<.->{\tiny
Widget w1;
Widget w2\{w1\};}\end{codeblock}

  \item [copy assignment] Allows to change the value of an \alert{existing} object as a copy of another object

    \begin{codeblock}<.->{\tiny
Widget w1, w2;
w2 = w1;}\end{codeblock}

  \end{description}

  \begin{itemize}[<3->]
  \item The two objects are/remain distinct
  \item The copied-from object is not changed
  \item After the copy the two objects should compare equal
  \end{itemize}

\end{frame}

\begin{frame}[fragile]{Move operations}

  \begin{codeblock}{\tiny
class Widget \{
  \ddd
  Widget(Widget&& other);
  Widget& operator=(Widget&& other);
\};}\end{codeblock}

  \begin{description}[<2->]
  \item [move constructor] Allows the \alert{construction} of an object stealing
    the internals of another object

    \begin{codeblock}<.->{\tiny
Widget w\{make\_widget()\};}\end{codeblock}

  \item [move assignment] Allows to change the value of an \alert{existing}
    object stealing the internals of another object

    \begin{codeblock}<.->{\tiny
Widget w;
w = make\_widget();}\end{codeblock}

  \end{description}

  \begin{itemize}[<3->]
  \item The two objects are/remain distinct
  \item The moved-from object is usually changed
    \begin{itemize}[<*>]
    \item to a \textit{valid but unspecified} state
    \item it must be at least destructible and possibly reassignable
    \end{itemize}
  \end{itemize}

\end{frame}

\begin{frame}[fragile]{On move}
  \begin{itemize}
  \item<1-> A move is typically cheaper than a copy, but it can be as expensive
  \item<2-> If the \textit{Return Value Optimization} is not applied, the return
    value of a function is moved, not copied, into destination
  \item<3-> \code{operator=(T\&\&)} can assume that the argument is a temporary,
    hence different from \code{this}
    \begin{itemize}
    \item There is no need to check for self-assignment
    \item But be sure that in such event there is no crash
    \item Rule of thumb: \code{std::swap} must work

      \begin{codeblock}
template<typename T>
void swap(T& a, T& b) \{
  T t\{std::move(a)\};
  a = std::move(b);
  b = std::move(t);
\}\end{codeblock}

    \end{itemize}
  \end{itemize}
\end{frame}

\begin{frame}[fragile]{\code{= default}}
  \begin{itemize}
  \item Explicitly tell the compiler to generate a \underline{special member
    function} according to the default implementation
  \end{itemize}

  \begin{codeblock}<2->
class Widget \{
  int i = 0;
 public:
  Widget(Widget const&);
  \uncover<3->{Widget() \alert<3>{= default};}
\};

\uncover<2->{static\_assert(std::is\_copy\_constructible<Widget>::value);}
\uncover<2->{static\_assert(\only<2>{!}std::is\_default\_constructible<Widget>::value);}\end{codeblock}

\end{frame}

\begin{frame}[fragile]{\code{= delete}}

  \begin{itemize}
  \item<1-> A function can be declared as \textit{deleted}, marking it with
    \mbox{\alert{= delete}}
  \item<2-> For example, a class can be made \alert{non copyable} deleting its
    copy operations
  \item<3-> Calling a deleted functions causes a compilation error
  \item<5-> Any function can be deleted
  \end{itemize}

  \begin{codeblock}<1->
template<typename P>
class SmartPointer \{
  \ddd
  \textbf<3>{SmartPointer(SmartPointer const&) \alert<1>{= delete};}
  \textbf<4>{SmartPointer& operator=(SmartPointer const&) \alert<1>{= delete};}
\};

\uncover<2->{using SPI = SmartPointer<int>;

static\_assert(!std::is\_copy\_constructible<SPI>::value);
static\_assert(!std::is\_copy\_assignable<SPI>::value);}

\uncover<3->{SPI sp1, sp2;}
\uncover<3->{SPI sp3\{sp1\}; // error}
\uncover<4->{sp2 = sp1;    // error}\end{codeblock}

\end{frame}

\section*{Error management}

\begin{frame}{Mechanisms for error management}

  \centering\alert{The sooner the errors are identified, the better}

  \begin{itemize}
  \item \texttt{static\_assert}
    \begin{itemize}
    \item Logical assertion that must be valid at \underline{compile time}
    \end{itemize}
  \item \texttt{assert}
    \begin{itemize}
    \item Logical assertion that must be valid at \underline{run time}
    \end{itemize}
  \item Exceptions
    \begin{itemize}
    \item To express an error condition happening at \underline{run time},
      typically related to a lack of resource
    \end{itemize}
    \item \color[rgb]{.6,.6,.6}{C-style error codes}
      \begin{itemize}
      \item \color[rgb]{.6,.6,.6}{They can be ignored (but they should not!)}
      \end{itemize}
    \item \ldots
  \end{itemize}
\end{frame}

\begin{frame}[fragile]{\texttt{static\_assert}}
  Check that a certain constant boolean expression is satisfied during
  compilation
  \begin{itemize}
  \item If not, fail compilation with the specified message
  \end{itemize}

  \begin{codeblock}<2->{
#include <type\_traits>

struct C \{
  C(C const\&) = default;
  C\& operator=(C const\&) = delete;
\};

\alert<2-3>{static\_assert(!std::is\_default\_constructible<C>::value, "");
static\_assert( std::is\_copy\_constructible\_v<C>);
static\_assert(!std::is\_copy\_assignable\_v<C>);
static\_assert( std::is\_\only<3->{nothrow\_}move\_constructible\_v<C>);
static\_assert(!std::is\_move\_assignable\_v<C>);
static\_assert( std::is\_destructible\_v<C>);
static\_assert(sizeof(C) == 1);}}\end{codeblock}

  \uncover<4->{
    A static assertion declaration can appear practically anywhere
    \begin{itemize}
    \item There is no effect, hence no overhead, at run time
    \end{itemize}
  }
\end{frame}

\begin{frame}[fragile]{\texttt{assert}}

  Check that a certain boolean expression is satisfied at run time
  \begin{itemize}
  \item If not satisfied, it means that the state of the program is corrupted
    $\rightarrow$ better to close the program as soon as possible (calling
    \code{std::abort})
  \end{itemize}

  \begin{codeblock}<2->{
template<class T> class Vector \{
  T* p;
  \ddd
  T\& operator[](int n) \{
    \uncover<3->{\alert{assert(p != nullptr);         // class invariant (sort of)}}
    \uncover<4->{\alert{assert(n >= 0 \&\& n < size()); // function pre-condition}}
    return p[n];
  \}
\};}\end{codeblock}

  \uncover<5->{
    Useful during testing/debugging
    \begin{itemize}
    \item Can be disabled for performance reasons (\texttt{-DNDEBUG})
    \item Avoid side effects in \texttt{assert}s
    \end{itemize}
  }
\end{frame}

\begin{frame}[fragile]{Exceptions}

  \begin{itemize}
  \item Mechanism to report errors out of a function, stopping its execution
  \item Useful to express \textit{post-conditions}
  \item Help separate application logic from error management
  \end{itemize}

  \begin{codeblock}<2->{
class Thing \{\ddd\};
\uncover<5->{class Exception \{\ddd\};}

auto make\_thing() \{
  \uncover<3->{auto res = acquire\_resources\_to\_build\_thing();}
  \uncover<4->{if (!success(res)) \{}
    \uncover<5-6>{Exception e\{\ddd\};}
    \uncover<5->{throw \alt<-6>{e}{Exception\{\ddd\}};}
  \uncover<4->{\}}
  return Thing\{\uncover<3->{res}\}; \uncover<6->{// not executed in case of exception}
\}}\end{codeblock}

  \uncover<8->{Note that all local variables (e.g. \texttt{res}) are properly
    destroyed when exiting the function, be it via \texttt{return} or via
    \texttt{throw}}
\end{frame}

\begin{frame}[fragile]{Exception propagation}

  \begin{columns}[t]

    \begin{column}{.5\textwidth}
      \begin{codeblock}
\uncover<1->{auto high() \{}
\uncover<9->{  try \{}
\uncover<2->{    // this part is executed}
\uncover<1->{    mid();}
\uncover<10->{    // this part is not executed}
\uncover<9->{  \} catch (E\alert<12>{\&} e) \{}\uncover<12->{ // \alert<12>{by reference}}
\uncover<11->{    // use e}
\uncover<9->{  \}}
\uncover<1->{\}}

\uncover<1->{auto mid() \{}
\uncover<3->{  T t; // this part is executed}
\uncover<1->{  low();}
\uncover<7->{  // this part is not executed}
\uncover<8->{  // \alert<8>{T is properly destroyed}}
\uncover<1->{\}}

\uncover<1->{auto low() \{}
\uncover<4->{  // this part is executed}
\uncover<5->{  throw E\{\};}
\uncover<6->{  // this part is not executed}
\uncover<1->{\}}\end{codeblock}
    \end{column}

    \begin{column}<13->{.5\textwidth}
      \begin{itemize}
      \item An exception is propagated up the stack of function calls until a
        suitable catch clause is found
      \item If no suitable catch clause is found the program is
        \texttt{terminate}d
      \item During stack unwinding all automatic objects are properly destroyed
        \begin{itemize}
        \item Remember smart pointers!
        \end{itemize}
      \end{itemize}
    \end{column}

  \end{columns}
\end{frame}

\begin{frame}{Exception safety}

  Different levels of safety guarantees (for member functions):

  \begin{description}
  \item [basic] If an exception is thrown, no resource is leaked and the object
    is left in a \textit{valid but unspecified} state
    \begin{itemize}
    \item the object should be at least safely assignable and destroyable
    \item every class should provide at least the basic guarantee
    \end{itemize}
  \item [strong] Transaction semantics: if an exception is thrown, the object's
    state is as it was before the function was called
  \item [no-throw] The operation is always successful and no exception leaves
    the function
  \end{description}

\end{frame}

\begin{frame}[fragile]{\texttt{noexcept}}

  \begin{itemize}
  \item A function can be declared \texttt{noexcept}, telling the compiler that
    the function
    \begin{itemize}
    \item doesn't throw, or
    \item<2-> is not able to manage exceptions \uncover<3->{$\rightarrow$ better
      \texttt{terminate}}
    \end{itemize}

    \begin{codeblock}<4->{
class Handle \{
  Handle(Handle\&\& o) \alert{noexcept} : \ddd \{ \ddd \}
  \ddd
\};}\end{codeblock}

  \item<5-> Declaring functions (not only member functions) \texttt{noexcept} helps
    the compiler to optimize the code
  \item<5-> If move operations, especially the constructor, are \texttt{noexcept} the
    compiler/library can apply \textbf{significant} optimizations
    \begin{itemize}
    \item E.g. in order to provide the strong guarantee
      \texttt{std::vector::push\_back} must copy, not move, objects, if the move
      can throw
    \end{itemize}
  \end{itemize}

\end{frame}

\begin{frame}{Move and \texttt{noexcept}}
  \begin{itemize}
  \item<1-> \texttt{T\& T::operator=(T\&\& tmp)} is typically easy to make
    \texttt{noexcept}
    \begin{itemize}
    \item<2-> Rely on the \texttt{noexcept}-ness of data members' move-assignments
    \end{itemize}
  \item<3-> \texttt{T::T(T\&\& tmp)} may be more difficult
    \begin{itemize}
    \item<3-> Start with one object (\texttt{tmp}), end up with two (\texttt{*this}
      and \texttt{tmp})
    \item<4-> Can rely on \texttt{T::T()} being \texttt{noexcept} as well
    \item<4-> Which is not obvious if a resource has to be acquired
    \end{itemize}
  \end{itemize}
\end{frame}

\begin{frame}[fragile]{Destructor and \texttt{noexcept}}

  \begin{itemize}
  \item The destructor is by default \texttt{noexcept}
    \begin{itemize}
    \item i.e. releasing a resource should not fail
    \end{itemize}
  \item<2-> Don't do anything overly complicated in the destructor or swallow all
    exceptions locally
  \end{itemize}

  \begin{codeblock}<3->
class Thing \{
  ~Thing()
  \{
    try \{
      \vdots
    \} catch (\alert{...}) \{ // catch \alert{all} exceptions
      // e.g. log something, provided logging doesn't throw
    \}
  \}
\};\end{codeblock}

  \begin{itemize}
  \item<4-> It's always possible to declare a destructor, like any other function,
    \texttt{noexcept(false)}
  \end{itemize}

\end{frame}
